\documentclass[a4paper,11pt]{article}

\usepackage{listings}
\usepackage{verbatim}

\newcommand{\tcode}[1]{\texttt{#1}}

\title{The CTwist, a Twisted C Language and Library Specification to take programming to infinity}
\date{in progress}
\author{Ruben Van Boxem}

\begin{document}

\maketitle

\newpage

\tableofcontents
\thispagestyle{empty}
\newpage

\section{Introduction}

CTwist aka Twisted C, is a language inspired by the rich featureset of C++.
It is an attempt to simplify most incredibly daft points in the currect C++ language, clone the most useful features, and improve those features that were badly or nonuniformly designed.
Although the aim is feature parity or even expansion, the syntax will be radically different in order to simplify the compilation process and improve the readability of previously written code.
Not only will the library’s interface be more uniform, the archaic C standard library will be placed outside of CTwist’s world, being only callable as a foreign interface.
The Twisted C Standard Library will replace all necessary functionality contained in the C Standard library, ensuring a uniform callable interface without mixing inherent language style and intent.
A shortlist of the most prominent features of Twisted C are listed below.

\begin{itemize}
  \item Strong, static typing
  \item Inheritable fundamental types
  \item Strong typedefs
  \item Classes
  \item Parametric polymorphism (often referred to as generics or templates)
  \item Low level memory access (raw pointers to arbitrary memory locations)
  \item Exceptions
  \item No pre-processor
  \item ABI standardization (to the level of CTwist object code, different OSes enforce different low-level ABI’s CTwist cannot change)
  \item Unicode support, inside and out.
  \item Integrated multithreading in basic language constructs.
  \item A large Standard library, including a lot of functionality that has been wrapped over and over by different C and C++ libraries like Boost, Qt, etc...
  \item Type inference?
  \item User-defined operators?
  \item Integrated documentation generation (to replace readable headers)?
\end{itemize}
Example programs with a tutorial-like explanation can be found in section \ref{sec:examples}, displaying typical syntax to express common programming constructs.

\section{Files and environment}

In an attempt to bring order to the chaos that C-like languages face and cause a lot of grief for programmers, CTwist defines the following strict rules to be followed in all files involved in CTwist source and object code.

  \subsection{Source files}

Source files will be encoded in UTF-8 to avoid byte order issues or obsolete system encodings being used.
Both \verb \ \tcode{n} and \verb \ \tcode{r} \verb \ \tcode{n} will be valid and must be accepted interchangeable by the compiler (different files may have different line endings, one line ending type per file).
\\ ...

  \subsection{Compilation model}

The reference compiler is implemented on top of the LLVM infrastructure, and therefore the reference compilation model will base itself upon the LLVM bytecode representation, which is a flexible and platform-independent compiled and perhaps optimized intermediary representation, allowing for easy link-time optimization (through LLVM machinery).
Alongside the LLVM bytecode, there will be a binary representation of “generic” code, necessary to provide C++ template functionality without a “header” concept.
Essentially, a compiled “Twisted Object” file will consist of the following:
\begin{itemize}
  \item Binary (compiler) representation of function and class declarations (the compiler’s Abstract Syntax Tree)
  \item Symbol table to match the declarations to the LLVM bytecode.
  \item Binary (compiler) representation of generics
  \item LLVM bytecode for non-generic functions (specializations).
\end{itemize}
The binary representation will be in such a form that it can be quickly loaded by the compiler on demand.
The LLVM bytecode is only present for the purpose of producing a final binary, and not used in compiling source code.


\section{Program structure}


\section{Examples}
This section contains expressive and classic examples of programs and constructs often employed in tutorial or real programs and libraries.

  \subsection{Beginner programs}

    \subsubsection{“Hello World!”}

\begin{lstlisting}
entry
{
    std.io.print( “Hello World!” )
}
\end{lstlisting}

\end{document}