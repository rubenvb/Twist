\documentclass[a4paper,11pt]{report}

\usepackage{geometry}
\geometry{
  includeheadfoot,
  margin=2.54cm
}

\usepackage{listings}
\lstset{basicstyle=\ttfamily,numbers=left,numberstyle=\tiny,}

\usepackage{multicol}
\usepackage[colorlinks]{hyperref}
\hypersetup{linkcolor=[rgb]{0.2,0.4,0.7}} % light blue link color\hyper
\usepackage{verbatim}

\newcommand{\tcode}[1]{\texttt{#1}}
\newcommand{\eg}{\textit{e.g.}}

\title{Twist, the twisted programming language\\Environment, language, and Library Specification}
\date{in progress}
\author{Ruben Van Boxem}

\begin{document}

\maketitle

\tableofcontents
\thispagestyle{empty}

\chapter{Introduction}

Twist, is a language inspired by the rich feature set of C++, unencumbered by syntactical compatibility with C and the syntactic awkwardness of some C++ constructs like templates.
It is an attempt to simplify most incredibly daft points in the current C++ language, clone the most useful features, and improve those features that were badly or non-uniformly designed.
Although the aim is feature parity and even expansion, the syntax will be radically different in order to simplify the compilation process, improve code readability, and enhance application development speed.
Not only will the library's interface be more uniform, the archaic C standard library will be placed outside of Twist’s world, being only accessible as a foreign language interface, if you need some library written in C (as there are many).
For this reason, and the basic low-level interoperation with the OS interfaces, the fundamental types in Twist will map transparently to the fundamental C types.
The Twisted C Standard Library (TCSL) will replace all necessary functionality contained in the POSIX\footnote{IEEE Std 1003.1\texttrademark-2008, available online at \url{http://pubs.opengroup.org/onlinepubs/9699919799/} and part of the Single Unix Specification v4 available at \url{http://www.unix.org/version4/}} standard, ensuring a uniform callable interface without the need for the user to worry about the underlying platform.
A shortlist of the most prominent features of Twisted C are listed below.

\begin{itemize}
  \item No pre-processor,
  \item Static typing with type inference,
  \item Strong typedefs and weak aliases,
  \item User-defined objects,
  \item Generic constructs,
  \item Low level memory access (raw read-write to arbitrary locations in memory in the program's address space),
  \item Exceptions,
  \item Standardized name mangling and low-level-runtime library.
  \item Unicode support, inside and out (this ``out" will take a while to complete...),
  \item Integrated multithreading in basic constructs like algorithms and control structures,
  \item A full Standard library.
  \item Integrated flexible documentation system
\end{itemize}
Example programs with a tutorial-like explanation can be found in section \ref{sec:examples}, displaying typical syntax to express simple examples.

\chapter{Environment}

Twist relies on a system interface (\eg the system's C library).
This environment should provide basic math, I/O and other system interfaces, and some form of low-level exception handling.
This chapter is concerned with strictly defining the environment that a Twist compiler operates in.

\section{Compilation}

A Twist compiler shall take Twist \hyperref[sec:source_files]{\emph{source files}}, and produce Twist \hyperref[sec:object_files]{\emph{object files}}.
These object files can be \hyperref[sec:linking]{linked} into a final \hyperref[sec:shared_objects]{shared object} or \hyperref[sec:executables]{executable} described in a \hyperref[sec:module_description_files]{module description file}.

TODO: explain the actual compilation process, needs explanation of module system.

\section{Twist files}

\subsubsection{Source files} \label{sec:source_files}

The textual source code input for a Twist compiler can be from a file or other implementation defined I/O device (\eg standard input).
Source files shall be named \tcode{\textit{filename}.t}.
Source files shall be encoded in UTF-8.

\subsubsection{Object files} \label{sec:object_files}

Object files contain everything needed by the compiler, linker and documentation system to work efficiently.
Object files shall be named \tcode{\textit{filename}.to}.
Object files shall be binary files.
All plain text contained
The code is translated into a mostly platform independent representation to account for generics and quick compilation.

\subsubsection{Module description files} \label{sec:module_description_files}


\section{Final output}




\section{Files and environment}

In an attempt to bring order to the chaos that C-like languages face and cause a lot of grief for programmers, Twist defines the following strict rules to be followed in all files involved in Twist source and object code.
All text will be encoded in UTF-8\footnote{See the Unicode Standard, Chapter 2} to avoid byte order issues or obsolete system encodings being used.
The presence of a Byte Order Mark (BOM)\footnote{The UTF-8 BOM in hexadecimal notation is the byte sequence \tcode{EF BB BF}: some platform’s editors add this} is allowed, but not required.

  \subsection{Module description files}

Module description files are nothing more than a list of source files to be compiled at once into a Twisted Module.
This avoids two platform-specific problems:
\begin{enumerate}
  \item commandline length limit if there are a lot of files,
  \item unicode filenames that may nor be easily usable on the commandline.
\end{enumerate}
The file has the following format:
\begin{lstlisting}
file1.ct
file2.ct
directory/file3.ct
../otherdirectory/file4.ct
\end{lstlisting}
Although it is discouraged to use \tcode{..} for module files, it may not be possible everywhere.
The directories are searched relative to the location of the module description file.
In the simplest case, one can also use the following form:
\begin{lstlisting}
*.ct
\end{lstlisting}
The \tcode{*} is a universal wildcard, which can be replaced by any number of any characters (except the directory seperator \tcode{/}), thus gathering all source files in a simple and readable manner.

  \subsection{Source files}

Source files are plain text files containing no hidden binary data.
Both \verb \ \tcode{n} and \verb \ \tcode{r} \verb \ \tcode{n} line endings will be valid and must be accepted interchangeable by the compiler (different files may have different line endings, one line ending type per file).

  \subsection{Binary files}

Binary files can take two forms: native and intermediate.
Native files contain only architecture/OS-native code that can be executed or loaded by the targetted OS.
They are meant for end-user execution, not development.
Intermediate files are still platform-independent and contain a platform-independent compiled representation of the source code.
These are meant to be used for development, and provide a means to quickly test minor changes without recompiling and linking everything.
A combination of both is possible as well, and handy only for increasing development speed (optimizations can be done once, and the resulting code is reused from then on).
So "native" and "intermediate" only affects the type of compiled code, and does not say anything about generic code presence.

  \subsection{Native Binaries}

Native binaries do not contain generic code.
Examples include executables, Windows DLL's, Linux' \tcode{.so} files, Mac OS X' \tcode{.dylib} files.
In general, these binary files only depend on core OS components, and any Twisted C libraries they use.
Optionally, legacy C code can be linked in through a compatibility interface \ref{sec:c_compatibility}, which is also used to call into OS APIs in a transparent fashion.

  \subsection{Intermediate files}

Intermediate files contain everything the Twist compiler needs to generate native binaries.
Nothing prevents an implementation to only provide a compiler AST representation in an intermediate file.
This will allow for all forms of syntax and well-formedness checking, but is a bit inefficient to debug exection (code will have to be generated over and over).
Along with AST representation of code, LLVM bytecode of non-generic constructs can be present.
This bytecode can be pre-optimized, to increase the code generation phase. 

  \subsection{Hybrid files}

Hybrid files contain non-native code and 

  \subsection{Compilation model}
    \subsubsection{Phases of translation}

Translation is the process in which a Twist compiler converts source code input into a Twisted Object file or 

    \subsubsection{Reference implementation}

The reference compiler is implemented on top of the LLVM infrastructure, and therefore the reference compilation model will base itself upon the LLVM bytecode representation, which is a flexible and platform-independent compiled and perhaps optimized intermediary representation, allowing for easy link-time optimization (through LLVM machinery).
Alongside the LLVM bytecode, there will be a binary representation of “generic” code, necessary to provide C++ template functionality without a “header” concept.
Essentially, a compiled “Twisted Object” file will consist of the following:
\begin{itemize}
  \item Binary (compiler) representation of functions and classes declarations, including the defini (the compiler’s Abstract Syntax Tree)
  \item Symbol table to match the declarations to the LLVM bytecode.
  \item LLVM bytecode for non-generic functions and fully specialized generic functions.
\end{itemize}
The binary representation will be in such a form that it can be quickly loaded by the compiler when needed.
It contains all information needed to resolve parametric polymorphic resolution, both for functions and classes.
The binary representation will be open and fully specified, with reference tools to verify and test the validity.
\\ ...

\section{Program structure}
  \subsection{Introductory note}


  \subsection{Entry point}

Any Twisted Object file can contain a formal \emph{entry point}.
The result is that multiple Twisted Objects with entry points defined can be linked together. 

\section{C compatibility} \label{sec:c_compatibility}

Most if, not all, OS APIs are written in C.
Twist, as a flexible compiled language, has an easy interface to C libraries.
The basic C integer types and pointers to structs are allowed in C interface declarations.
Other additional keywords are ignored and it is the task of the interface user to ensure the Twist wrapper ensures the calls to the external functions are valid.
As a simple example, the following code fragment declares an external C function in the reserved namespace C:
\begin{lstlisting}
namespace C.libraryname
{
  void cfunction(int, char*, size_t, struct cstruct*)
}
\end{lstlisting}
The C types recognized are listed in Appendix \ref{app:c_types}.

\section{Twisted Standard Library -  defer to "French KISS" reference implementation}
  \subsection{Components}

Below is the list of Library Components in the Twisted Standard Library.
\begin{itemize}
  \item Algorithms: basic, widely applicable algorithms.
  \item Input/Output: I/O to/from numerous devices like GUI, files, and console.
  \item Math: extensive mathematical functionality framework
  \begin{itemize}
    \item Basic: basic math functions like \tcode{exp}, \tcode{sin}, \tcode{sqrt}, etc...
    \item Extended: Advanced math functions, in general terms derived from the Hypergeometric function, like \tcode{besselj}, \tcode{erf}, etc...
    \item String to function parsing, extendible with custom operators.
    \item Integration: Basic numerical integration routines.
  \end{itemize}
  \item Containers: various useful container classes
  \item Strings: UTF-8/16/32 strings and associated routines
  \item Concurrency
  \begin{itemize}
    \item Threads
    \item Mutex and other Concurrent control objects and functions...
  \end{itemize}
  \item File system: abstract view of the system's file system, useful to manipulate directory paths and open files.
  \item Signals
  \item GUI
  \item ...
\end{itemize}
Below is a detailed description of what each Component contains and how different Components can work together.
TODO

\section{Examples} \label{sec:examples}
This section contains expressive and classic examples of programs and constructs often employed in tutorial or real programs and libraries.

  \subsection{Beginner-level examples}

    \subsubsection{``Hello World!"}

The classic first program a programmer meets whe learning a new language is without a doubt ``Hello World!".
It presents the minimal program structure and introduces a way to do textual output.
\begin{lstlisting}
entry
{
  std.io.print( "Hello World!" )
}
\end{lstlisting}
\lstinline|entry| is Twist's version of the \lstinline|main| function in languages like C, C++, C\# or Java.
In contrast to other, more primitive languages, it takes no arguments under any circumstances.
It has no return type, and is not strictly a function that could be called in a devious manner.
A code block is surrounded by \tcode{\{} and \tcode{\}} and is used to designate several lines of code are to be seen as one entity.
A Twisted C statement does not have to end in a semicolon ``;", but it may.
A semicolon is useful if you want to stuff more than one statement on one line of code.
\tcode{std.io.output} is the fully qualified call to the \tcode{output} function in namespace \tcode{io} in namespace \tcode{std}, which designates the I/O Standard Library component.
The \tcode{std.io.output} function outputs to the Standard Output Device, which can be reseated by the programmer if deemed necessary.
An alternative version using the \tcode{import} keyword is written below, allowing the function call to be significantly simplified:
\begin{lstlisting}
import std.io
entry
{
    output( "Hello World!" )
}
\end{lstlisting}
Note that in building this program, absolutely no libraries need to be manually linked when producing an executable (they're not even mentioned "under the hood").
The \tcode{import} statement ensures that the resulting Twisted Object file contains a directive to pull in the \tcode{std.io} Library Component.
It also functions as a C-style \tcode{\#include}, pulling in all the functions and types in the Component \tcode{std.io}.

    \subsubsection{``Hello World!" in a file}

Instead of outputting the string to the Standard Output Device, the following program writes it to a file named \tcode{Hello World!.txt}.
\begin{lstlisting}
import std.io
entry
{
    file f = file.open( "Hello World!.txt" )
    f.output( "Hello World!" )
    f.close
}
\end{lstlisting}
In this program, \tcode{std.io} has once again been imported.
It includes basic file I/O, and a class called \tcode{file}, which (not surprisingly) represents a file in Twisted C.
The last line is not strictly necessary, as the file will get closed automatically when the object \tcode{f} goes out of scope at the closing \tcode{\}}.

    \subsubsection{``Hello From Thread!"}

The following program demonstrates the simplest of Concurrency features: a thread of execution.

  \subsection{Intermediate examples}
    \subsubsection{Calling C code}

As described in Section \ref{sec:c_compatibility}

  \subsection{Advanced examples}
    \subsubsection{None yet}

\appendix
\section{Reserved file extensions}
Table \ref{tab:file_extensions} is a comprehensive list of file extensions reserved and usable by the Twist implementation.
\begin{table}[h] \label{tab:file_extensions}
\begin{tabular}{r|l}
.?? & Module Description File: list of files to be compiled into one Twisted Object file. \\
.ct & Source file \\
.to & Twisted Object file: contains binary, linkable, JIT-able code
\end{tabular}
\end{table}

\section{C compatibility}
\subsection{ABI}

\subsection{C Types} \label{app:c_types}
Within \tcode{namespace C} sections one can use several fundamental C types, listed below:
\begin{lstlisting}
char
short
wchar_t
int
long
long long
char16_t
char32_t
float
double
_Bool
_Complex
\end{lstlisting}
\tcode{unsigned} variants are also supported everywhere where they are legal in C.
Fixed-size types (integer \tcode{int[least]NN\_t} types) as first defined in C99 are supported natively, and using the Twist typenames in those cases will guarantee the correct underlying C type is used where there might be ambiguity across different platforms or architectures.

\end{document}